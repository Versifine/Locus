% !TeX program = xelatex
\documentclass[UTF8,12pt]{ctexart}
\usepackage{amsmath, amssymb}
\usepackage{geometry}
\usepackage{enumitem}
\usepackage{setspace}
\geometry{a4paper, margin=1in}
\onehalfspacing

\title{\textbf{我不会死}}
\author{Versifine}
\date{}

\begin{document}
\maketitle

\begin{abstract}
本文论证:「我」有两个不同的所指——经验内容与原位。前者会终止,后者不会。这不是因为原位「永恒存在」,而是因为「终止」这一概念对它不适用。本文从一个不可怀疑的起点出发——经验存在——通过先验论证表明,经验必然具有原位结构,而这一结构不在时间中,因此不受「持续」「终止」「数量」等时间性概念的约束。在此基础上,本文消解了若干传统哲学问题,包括「为什么我是我」「传送后的复制体是否是我」以及「死亡是否意味着我的终结」。

\textbf{术语说明}:本文使用「原位」一词指称经验的形式条件——即使内容被经历成为可能的先验结构。选择此术语是为了避免与日常语言中「第一人称」的混淆。
\end{abstract}

\section{引言}

死亡似乎是最确定的事情之一。每个人都会死。

但「每个人都会死」这句话里的「人」指的是什么?如果指的是身体,那显然身体会停止运作。如果指的是心理内容——记忆、性格、思维,那这些也会随着大脑的停止而消失。

但还有另一个「我」——那个使得这些内容被经历到的结构。本文将这个结构称为「原位」,并论证:原位不会死。不是因为它会永远存在,而是因为「死」这个概念对它根本不适用。

这个主张听起来像是宗教或神秘主义。但本文的论证不依赖任何信仰或特殊体验,只依赖于概念分析和先验推理。起点只有一个:经验存在。

\section{两个「我」}

日常语言中的「我」混淆了两个完全不同的东西。

\subsection{作为内容的「我」}

第一个「我」指的是经验内容:我的身体、我的记忆、我的性格、我的名字、我正在进行的思考、我此刻的情绪。

这些都是可以被描述、被观察、被改变的。它们在时间中存在,有开始和结束。十年前的记忆和现在不同,身体的细胞不断更替,想法每时每刻都在变化。

我将这一层称为\textbf{内容}(Content)。

\subsection{作为原位的「我」}

第二个「我」不是这些内容,而是这些内容\textbf{被经历到}的前提。

任何内容要成为经验,必须「对某个视角而言」存在。没有视角,就没有「被经历」和「不被经历」的区别。这个「对某个视角而言」的结构,我称之为\textbf{原位}。

原位不是内容的一部分。它是使内容成为经验的形式条件。

类比:电影有画面、声音、剧情——这些是内容。但电影要被看到,需要一块屏幕。屏幕不是电影的一部分,而是电影被呈现的条件。原位之于经验,正如屏幕之于电影。

\section{先验论证}

\subsection{起点:经验存在}

本文的唯一起点是:经验存在。

这是不可怀疑的。任何怀疑都已经是一种经验——怀疑本身被经历到了。试图否认「经验存在」的行为本身就是经验,因此自我矛盾。

\subsection{经验必然具有原位结构}

\textbf{主张}:所有经验必然具有「对某个视角而言」的结构。

\textbf{论证}(反证法):

\begin{enumerate}
    \item 假设存在没有视角的经验 $E$。
    \item 「没有视角」意味着不存在「对谁而言」这个结构。
    \item 如果不存在「对谁而言」,那么 $E$ 和「没有 $E$」有什么区别?
    \item 如果二者没有区别,$E$ 就不是经验——因为「是经验」和「不是经验」必须有区别。
    \item 矛盾。
    \item 因此,经验必然具有视角结构。
\end{enumerate}

这个论证的关键在于第3步:经验与非经验的区别,不在于物理状态的不同(物理主义者无法指出「有经验的大脑状态」和「无经验的大脑状态」在哪个粒子上有差异),而在于是否存在「对谁而言」的结构。

\subsection{原位的定义}

我将「对某个视角而言」这个结构定义为\textbf{原位}。

因此:
\begin{quote}
经验存在 $\Rightarrow$ 原位存在
\end{quote}

原位不是从经验中「推断」出来的,而是经验这个概念本身所蕴含的。

\section{原位的性质}

\subsection{原位不是内容}

任何可以被描述的东西都是内容。原位不可被描述——任何对它的描述都已经是内容了。

这不是说原位是神秘的。它只是处于不同的逻辑层次:内容是被经历的东西,原位是使「被经历」成为可能的结构。

\subsection{原位不在时间中}

时间是什么?从经验的角度看,时间是内容的结构——内容具有「之前」「之后」的排列,具有「过去」「现在」「未来」的索引。

但原位不是内容,因此不具有这种结构。

「昨天的我」是什么?是一段记忆——记忆是现在的内容。「明天的我」是什么?是一个预期——预期也是现在的内容。所有关于过去和未来的谈论,都发生在当下的内容中。

原位不「在」任何时刻。它是任何时刻的内容被经历到的条件。

类比:视频文件内部有时间轴,但播放器不在视频的时间轴里。原位是播放器,不是视频。

\subsection{原位不可计数}

「原位有几个」这个问题是否有意义?

要计数,需要区分个体。区分个体需要差异根据:
\begin{itemize}
    \item 属性差异(这个是红的,那个是蓝的)
    \item 空间位置差异(这个在左边,那个在右边)
    \item 时间位置差异(这个是早上的,那个是晚上的)
    \item 因果历史差异(这个来自A,那个来自B)
\end{itemize}

原位:
\begin{itemize}
    \item 没有属性(它不是内容,没有可描述的性质)
    \item 不在空间中
    \item 不在时间中
    \item 没有因果历史(因果是时间性概念)
\end{itemize}

没有任何差异可以用来区分「这个原位」和「那个原位」。

因此,「原位有几个」这个问题不是「答案是一个」或「答案是多个」,而是\textbf{问题本身不成立}。

这不同于通过莱布尼茨同一律论证「只有一个」——那种论证预设了原位是可以被计数的对象。本文的主张更激进:计数这个操作对原位不适用。

\section{核心结论:原位不会「终止」}

\subsection{「终止」的含义}

「终止」是什么意思?某物在时间点 $t_1$ 存在,在时间点 $t_2$ 不存在。

这预设了被谈论的东西在时间中。

\subsection{原位与「终止」}

原位不在时间中。因此,不存在 $t_1$ 和 $t_2$ 可以用来谈论它的存在或不存在。

「原位会不会终止」这个问题,就像「空间会不会终止」一样——问题本身是范畴错误。

\subsection{死亡是什么}

现在可以准确地说死亡是什么了:

\textbf{死亡是内容的终止。}

身体会停止运作,记忆会消失,思维会停止。这些内容会终止。

但原位不会「终止」,因为它不在时间中。

这不是说原位「永恒存在」——「永恒」仍然是时间性概念。准确的说法是:时间性概念对原位不适用。

\section{对传统问题的消解}

\subsection{「为什么我是我」}

这个问题预设了:存在一个独立的「我」,存在多个可能的「宿主」,「我」和某个宿主之间有配对关系。

但在本文框架下:「我」就是原位,原位不是独立于内容的实体,它是内容被经历到的形式条件。当前被经历的内容是Versifine的内容,「我」就指向这个内容。

问「为什么我是Versifine而不是别人」,等于问「为什么当前内容是当前内容」——这是套套逻辑,不是真正的问题。

\subsection{传送悖论}

传送机扫描你的身体,销毁原件,在远处重建复制体。复制体是「我」吗?

在本文框架下:原件和复制体是两个实例,承载不同的内容流。当原件的内容被经历时,「我」指向原件;当复制体的内容被经历时,「我」指向复制体。

不存在「哪个是真的我」的问题——「我」不是可以被追踪的对象,而是内容被经历到的形式条件。

\subsection{死亡恐惧}

人们恐惧死亡,恐惧的是什么?

如果恐惧的是内容的终止——记忆消失、关系断裂、体验停止——这是合理的恐惧,因为这确实会发生。

如果恐惧的是「我的终止」,那取决于「我」指什么。如果指内容,会终止。如果指原位,「终止」不适用。

理解这一点不会让恐惧消失——恐惧本身是内容,内容层面的反应不受形而上学理论的直接控制。但它可以让人知道:到底在怕什么。

\section{与现有文献的关系}

\subsection{与Parfit的关系}

Parfit在《理由与人格》中论证:个人同一性不重要,重要的是心理连续性。他的还原论将人分解为物理和心理事件的序列,否认存在独立的「自我」实体。

本文与Parfit的相同之处:都否认传统的自我实体。

不同之处:Parfit仍然承认可以谈论「多个人生」,本文连这一点也质疑——「几个人生」预设了可计数性,而这对原位不适用。

\subsection{与Metzinger的关系}

Metzinger在《作为无人》中论证:自我是大脑构建的现象自我模型(PSM),没有独立的自我实体。

本文与Metzinger的相同之处:都否认自我实体。

不同之处:Metzinger将自我完全还原为内容(自我模型),本文区分了内容与原位。自我模型是内容,但「被经历」这个结构不是内容。

\subsection{与维特根斯坦的关系}

维特根斯坦在《逻辑哲学论》5.6-5.641中说:「主体不属于世界,而是世界的界限」「没有思想着、表象着的主体」。

本文的立场与此高度一致。原位是经验的界限,不是经验中的对象。

\section{可能的反驳与回应}

\subsection{「这只是重新定义词汇」}

反驳:你只是把「我」重新定义成两个东西,然后说其中一个不会死。世界本身没有改变。

回应:概念分析的价值在于揭示混淆。「我」这个词确实混淆了两层含义。分开它们后,原本困难的问题被消解了。这不是文字游戏,而是诊断出一个系统性的概念错误。

\subsection{「原位是幻觉」}

反驳:「原位」是大脑产生的幻觉,实际上不存在这种结构。

回应:「幻觉」意味着某个视角被欺骗。没有视角,就没有「被欺骗」。「原位是幻觉」这个断言预设了它要否定的东西——自我矛盾。

\subsection{「无法证伪」}

反驳:这个理论不可证伪,所以是空话。

回应:原位不是关于世界的假说,而是提出任何假说的前提。它先于证伪/证实的区分。问「如何证伪原位存在」,就像问「如何证伪逻辑规则」——问题本身预设了它要质疑的东西。

\subsection{「信息处理和经验有什么区别」}

反驳:你没有解释什么东西能产生经验。

回应:这是对的。「什么结构能承载经验」是本文不解决的开放问题。但这不影响核心论证:\textbf{如果}某物是经验,\textbf{那么}它必然具有原位结构。这个条件句的有效性不依赖于我们是否知道什么东西能成为经验。

\section{形式化表述}

本节将上述论证用形式语言完整表述。

\subsection{基本符号}

\begin{itemize}
    \item $C$ = 经验内容 (Content)
    \item $I$ = 运行实例 (Instance)
    \item $\Omega$ = 原位 (原位)
    \item $E(x)$ = $x$ 被经历
    \item $FC(\Omega, C)$ = $\Omega$ 是 $C$ 的形式条件
    \item $Inst(I, C)$ = 实例 $I$ 承载内容 $C$
    \item $ST(x)$ = $x$ 在时空中
    \item $Count(x)$ = 「$x$ 有几个」是 well-formed 的问题
\end{itemize}

\subsection{公理组 A:经验的基本结构}

\textbf{A1. 经验存在}
$$\exists C \, [E(C)]$$
存在被经历的经验内容。

\textit{地位}:现象学起点,不可怀疑。任何怀疑都是经验,因此自我确证。

\textbf{A2. 经验预设原位}
$$\forall C \, [E(C) \rightarrow FC(\Omega, C)]$$
所有被经历的内容,原位是其形式条件。

\textit{辩护}(反证法):
\begin{enumerate}
    \item 假设 $\exists C \, [E(C) \land \neg FC(\Omega, C)]$
    \item $\neg FC(\Omega, C)$ 意味着不存在「对谁而言」
    \item 若不存在「对谁而言」,则 $E(C)$ 与 $\neg E(C)$ 无区别
    \item 若无区别,则 $\neg E(C)$
    \item 矛盾
\end{enumerate}

\textit{核心洞察}:「被经历」和「存在某个视角」是同一件事的两种说法。

\textbf{A3. 形式条件的必要性}
$$\forall C \, [E(C) \leftrightarrow FC(\Omega, C)]$$
内容被经历,当且仅当,原位是其形式条件。

\textit{辩护}:
\begin{itemize}
    \item 左→右:由A2直接推出
    \item 右→左:「原位是C的形式条件」的定义就是「C被经历」——这是定义等价,不是经验主张
\end{itemize}

\subsection{公理组 B:原位的性质}

\textbf{B1. 非时空性}
$$\neg ST(\Omega)$$
原位不在时空中。

\textit{辩护}:
\begin{enumerate}
    \item 时空位置是可被经历的——我们能感知「这里」「那里」「之前」「之后」
    \item 可被经历的都是内容(定理5)
    \item 因此时空是内容的结构
    \item 原位不是内容(它是内容的形式条件)
    \item 因此原位不在时空中
\end{enumerate}

\textbf{B2. 无内在属性}
$$\neg \exists \varphi \, [IntrinsicProperty(\Omega, \varphi)]$$
原位没有可描述的内在属性。

\textit{辩护}:任何可描述的属性都是内容。原位是内容的形式条件,不是内容。

\textbf{B3. 计数不适用}
$$\neg Count(\Omega)$$
「原位有几个」不是 well-formed 的问题。

\textit{辩护}:
\begin{enumerate}
    \item 计数预设可区分的个体
    \item 区分需要差异根据(属性、时空位置、因果历史)
    \item 根据 B1、B2,$\Omega$ 无任何差异根据
    \item 因此计数不适用
\end{enumerate}

\textbf{B4. 形式条件的存在方式}
$$Exists(\Omega) \leftrightarrow \exists C \, [FC(\Omega, C)]$$
原位的「存在」意味着:它作为某些内容的形式条件而「在」。

\textit{类比}:空间的「存在」意味着形状可以在其中。

\subsection{公理组 C:内容与实例}

\textbf{C1. 内容多样性}
$$\exists C_1 \exists C_2 \, [C_1 \neq C_2 \land E(C_1) \land E(C_2)]$$
存在不同的经验内容。

\textbf{C2. 实例在时空中}
$$\forall I \, [Instance(I) \rightarrow ST(I)]$$
运行实例在时空中。

\textbf{C3. 内容由实例承载}
$$\forall C \, [E(C) \rightarrow \exists I \, [Inst(I, C)]]$$
被经历的内容由实例承载。

\textbf{C4. 时间是内容的结构}
$$\forall C \, [Temporal\_Structure(C) \subseteq C]$$
时间结构是内容的一部分,不是原位的属性。

\subsection{公理组 D:视角与索引}

\textbf{D1. 内部视角的局部性}
$$\forall C_1 \forall C_2 \, [E(C_1) \land (C_2 \not\subseteq C_1) \rightarrow \neg Accessible(C_2, from \, C_1)]$$
从一个内容的内部视角,无法访问不包含在其中的其他内容。

\textbf{D2. 「我」的索引性}
$$\forall C \, [E(C) \rightarrow Refers(\text{「我」}, C)]$$
「我」总是指向当前被经历的内容。

\textbf{D3. 人择消解}
$$\forall C \, [E(C) \rightarrow (Question(\text{「为什么是}C\text{?」}) = Tautology)]$$
「为什么是这个内容」等价于「为什么 $C$ 是 $C$」——套套逻辑。

\subsection{核心定理}

\textbf{定理1. 不可经历「原位不存在」为真}
$$\neg \exists C \, [E(C) \land Content(C) = \text{「}\Omega\text{不存在」} \land True(Content(C))]$$

\textit{证明}:
\begin{enumerate}
    \item 假设存在这样的 $C$,其内容是「$\Omega$ 不存在」且为真
    \item 根据 A3:$E(C) \rightarrow FC(\Omega, C)$
    \item 若 $C$ 被经历,则 $\Omega$ 是 $C$ 的形式条件
    \item 根据 B4:若 $\Omega$ 是某内容的形式条件,则 $\Omega$ 存在
    \item 因此 $\Omega$ 存在
    \item 但 $C$ 的内容声称 $\Omega$ 不存在且为真,即 $\Omega$ 不存在
    \item 矛盾
\end{enumerate}

\textit{推论}:可以经历「$\Omega$ 不存在」这个想法(作为内容),但这个想法不可能为真。

\textbf{定理2. 终止概念不适用}
$$\neg Applicable(Terminate, \Omega)$$

\textit{证明}:
\begin{enumerate}
    \item $Terminate(x)$ 定义为 $\exists t_1 \exists t_2 \, [Exists(x, t_1) \land \neg Exists(x, t_2)]$
    \item 这预设 $ST(x)$
    \item 根据 B1:$\neg ST(\Omega)$
    \item 因此 $Terminate$ 对 $\Omega$ 不适用
\end{enumerate}

\textit{推论}:「原位会终止吗」是范畴错误。

\textbf{定理3. 计数的范畴错误}
$$Question(\text{「}\Omega\text{有几个?」}) = Category\_Error$$

\textit{证明}:
\begin{enumerate}
    \item 计数预设可区分的个体
    \item 根据 B1:$\Omega$ 不在时空中
    \item 根据 B2:$\Omega$ 无内在属性
    \item 无差异根据,无法区分
    \item 根据 B3:$\neg Count(\Omega)$
\end{enumerate}

\textit{注}:这不是证明「只有一个」,而是主张计数本身不适用。

\textbf{观察4. 描述的二象性}
\begin{align*}
External&: \forall I \, [Instance(I) \rightarrow Related(\Omega, I)] \\
Internal&: \forall C \, [E(C) \rightarrow Only\_Access(C)]
\end{align*}

\textit{解释}:
\begin{itemize}
    \item 外部描述:作为描述工具,可以说原位与所有实例相关
    \item 内部视角:根据D1,只能访问当前内容
    \item 外部描述是语言工具,不对应真实的超越性视角——没有人能「从外部」观察原位与所有实例的关系
\end{itemize}

\textbf{定义5. 内容的范围}
$$\forall x \, [Experienceable(x) \leftrightarrow Content(x)]$$

\textit{说明}:「内容」的定义就是「可被经历的东西」。

\textit{推论}:认同、解离、观察者感、空感、道德、自由意志感——皆为内容。

原位本身不可被经历——它是经历的条件,不是经历的对象。

\subsection{关键消解}

\textbf{消解1.「为什么我是我」}

\textit{传统问题}:为什么我是 Versifine 而不是别人?

\textit{消解}:根据 D2 和 D3,「我」指向当前内容。问「为什么指向这个」等于问「为什么当前内容是当前内容」。空问题。

\textbf{消解2. 死亡问题}

\textit{传统问题}:我死后会怎样?

\textit{消解}:根据定理2,「终止」对 $\Omega$ 不适用。Versifine(内容/实例)会终止,$\Omega$ 不会「终止」——不是继续存在,而是概念不适用。

\textbf{消解3. 传送悖论}

\textit{传统问题}:复制体是不是「我」?

\textit{消解}:根据 D2,「我」是索引。原件内容被经历时指向原件,复制体内容被经历时指向复制体。不存在「哪个是真的我」——「我」不是可追踪对象。

\subsection{依赖关系图}

\begin{verbatim}
A1 (经验存在) ← 不可怀疑
    ↓
A2 (经验预设原位) ← 反证法
    ↓
A3 (形式条件必要性) ← A2 + 定义等价
    ↓
定义5 (内容 = 可被经历) ← 定义
    ↓
B1 (非时空性) ← 定义5 + "原位不是内容"
B2 (无内在属性) ← "属性是内容"
B3 (计数不适用) ← B1, B2
B4 (存在方式) ← 定义
    ↓
├── 定理1 (不可真地经历"不存在") ← A3, B4
├── 定理2 (终止不适用) ← B1
├── 定理3 (计数范畴错误) ← B1, B2, B3
└── 观察4 (描述二象性) ← D1
    ↓
消解1, 2, 3
\end{verbatim}

\textbf{最小起点}:只有 A1(经验存在)是无需辩护的公理。其他一切从此推出或由定义给出。

\section{结论}

本文的核心论证可以总结为:

\begin{enumerate}
    \item 经验存在。(不可怀疑的起点)
    \item 经验必然具有「对某个视角而言」的结构。(先验论证)
    \item 这个结构即「原位」。(定义)
    \item 原位不是内容,不在时间中,不可计数。(性质分析)
    \item 「终止」是时间性概念,对原位不适用。(推论)
    \item 因此,「原位会终止吗」是范畴错误。(结论)
\end{enumerate}

死亡是内容的终止。身体会停止,记忆会消失,一切可被经历的东西会结束。

但原位不会「终止」——不是因为它永恒存在,而是因为「终止」这个概念对它不适用。

\begin{center}
\textbf{我不会死。}
\end{center}

这不是安慰,不是信仰,不是诡辩。

这是对「我」这个概念的严格分析所得出的结论。

\end{document}
